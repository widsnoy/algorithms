\documentclass[a4paper]{article}
\usepackage[UTF8]{ctex}
\usepackage[colorlinks,
linkcolor = black,
anchorcolor = blue,
urlcolor = blue,
citecolor = blue,
]{hyperref}
\usepackage{listings}
\usepackage{xcolor}
\usepackage{booktabs} % 导入三线表需要的宏包
\usepackage{longtable}% 导入跨页表格所需宏包
\usepackage{amsmath}
\usepackage{geometry}
\usepackage{fontspec}
\usepackage{fancyhdr}
\usepackage{graphicx}
\usepackage{amsfonts}
\usepackage{chemformula}
\usepackage{cite}
\usepackage{bookmark}
\usepackage{chemformula}
\usepackage[T1]{fontenc}
\setlength{\parindent}{0pt}

\geometry{left=2.0cm,right=1.5cm,top=2.0cm,bottom=2.0cm}
\pagestyle{fancy}
\fancyhead[L]{by widsnoy}
\fancyhead[R]{\enspace}
\fancyfoot[C]{\thepage}

\begin{document}
\fontsize{12pt}{22pt}\selectfont		
\title{\Huge \textbf{ICPC TEMPLATE}}
\author{widsnoy}
\maketitle
\pagenumbering{roman}
\tableofcontents
\newpage
\pagenumbering{arabic}

	
\lstset{
	breaklines=true,
	basicstyle=\fontspec{Fira Code}\zihao{5},
	keywordstyle=\fontspec{Fira Code SemiBold}\zihao{5},
	commentstyle=\fontspec{Noto Sans CJK SC}\zihao{5},
	extendedchars=false,                        % 解决代码跨页时,章节标题,页眉等汉字不显示的问题
	backgroundcolor=\color[rgb]{0.96,0.96,0.96},
	keywordstyle=\color{blue}\bfseries,        % 关键字颜色
	commentstyle=\color{green}, %[rgb]{0,0.6,0},          % 注释颜色
	stringstyle=\color{red},                  % 字符串颜色
	showstringspaces=false,                     % 不显示字符串内的空格
	numbers=left,                               % 显示行号
	frame=single,                               % 设置代码框形式
	tabsize=4, 
	language=c++,
	captionpos=t,                               % title在上方(在bottom即为b)
	rulecolor=\color[rgb]{0.8,0.8,0.8}          % 设置代码框颜色
}  
\tableofcontents

\section{动态规划}
\subsection{回退背包}
$dp[i][j]$ 表示选择 $i$ 个物品,体积为 $j$ 的方案数  
$f[i][j]$ 表示不考率其中某个物品,选择 $i$ 个物品,体积为 $j$ 的方案数  
$dp[i][j]=f[i-1][j-w]+f[i][j]\Rightarrow f[i][j]=dp[i][j]-f[i-1][j-w]$

\section{图论}
\subsection{最近公共祖先}
\lstinputlisting[]{./图论/最近公共祖先.cpp}

\subsection{2-SAT 前缀优化建图}

1. 当前点选择说明之前的前缀都未选择

之前的前缀选择说明当前点位被选择  

2. 之前的前缀选择说明当前前缀被选择

当前前缀未选择说明之前前缀未选择  

3. 当前点选择说明当前前缀选择  

当前前缀未选择说明当前点未选择

\subsection{Kruskal重构树的性质}
不难发现,原图中两个点之间的所有简单路径上最大边权的最小值 = 最小生成树上两个点之间的简单路径上的最大值 = Kruskal 重构树上两点之间的 LCA 的权值。

也就是说,到点 $x$ 的简单路径上最大边权的最小值 $\leq val$ 的所有点 $y$ 均在 Kruskal 重构树上的某一棵子树内,且恰好为该子树的所有叶子节点。

我们在 Kruskal 重构树上找到 $x$ 到根的路径上权值 $\leq val$ 的最浅的节点。显然这就是所有满足条件的节点所在的子树的根节点。

如果需要求原图中两个点之间的所有简单路径上最小边权的最大值,则在跑 Kruskal 的过程中按边权大到小的顺序加边。

\subsection{广义圆方树}
\begin{lstlisting}
int stk[N], n, m, top, cnt, low[N], dfn[N], dfc;
bool vis[N];
vector<int> G[N], T[N];

void tarjan(int u) {
    stk[++top] = u;
    low[u] = dfn[u] = ++dfc;
    for (int v : G[u]) {
        if (!dfn[v]) {
            tarjan(v);
            low[u] = min(low[u], low[v]);
            if (low[v] == dfn[u]) {
                cnt++;
                for (int x = 0; x != v; --top) {
                    x = stk[top];
                    T[cnt].push_back(x);
                    T[x].push_back(cnt);
                    val[cnt]++;
                }
                T[cnt].push_back(u);
                T[u].push_back(cnt);
                val[cnt]++;
            }
        } else low[u] = min(low[u], dfn[v]);
    }
}
int main() {
    cnt = n;
    for (int i = 1; i <= n; i++) if (!dfn[i]) {
        tarjan(i);
        --top;
    }
}
\end{lstlisting}

\subsection{点分树}

需要注意,点分树上的路径与原来的树完全没有关系。

\begin{lstlisting}
#include <bits/stdc++.h>
using namespace std;
using ll = long long;

const int mod = 998244353;
const int N = 2e5 + 5;

int n, q, val[N], rt, siz[N], SUM, mx[N];
int faz[N];
vector<int> T[N];
bool vis[N];

// 假设树剖已经写好

void getrt(int u, int fa) {
	siz[u] = 1;
	mx[u] = 0;
	for (int v : t.G[u]) if (v != fa && !vis[v]) {
		getrt(v, u);
		siz[u] += siz[v];
		mx[u] = max(mx[u], siz[v]);
	}
	mx[u] = max(mx[u], SUM - siz[u]);
	if (mx[u] < mx[rt]) rt = u;
}
void Dfs(int u, int fa) {
	vis[u] = 1;
	for (int v : t.G[u]) if (v != fa && !vis[v]) {
		rt = 0;
		SUM = siz[v];
		getrt(v, rt);
		T[u].push_back(rt);
		faz[rt] = u;
		Dfs(rt, rt);
	}
}

// 假设动态开点权值线段树已经写好

void dfs(int u, const int now) { // 预处理
	add(rt1[now], 0, n, dist(faz[now], u), val[u]);
	add(rt0[now], 0, n, dist(u, now), val[u]);
	for (int v : T[u]) dfs(v, now);
}
void upd(int u, int d) {
	int x = u;
	while (true) {
		add(rt1[u], 0, n, dist(x, faz[u]), d);
		add(rt0[u], 0, n, dist(x, u), d);
		if (u == root) break;
		u = faz[u];
	}
}
int qry(int u, int k) {
	int ans = query(rt0[u], 0, n, 0, k), x = u;
	while (u != root) {
		if (k - dist(faz[u], x) >= 0) ans += query(rt0[faz[u]], 0, n, 0, k - dist(faz[u], x)) - query(rt1[u], 0, n, 0, k - dist(faz[u], x));
		u = faz[u];
	}
	return ans;
}
\end{lstlisting}

\subsection{二分图}
\subsubsection*{最大匹配}
\lstinputlisting[]{./图论/二分图匹配.cpp}

\subsection{有向图最小路径覆盖问题}
\lstinputlisting[]{./图论/有向图最小路径覆盖.cpp}

\subsection{网络流}
\subsubsection{Dinic最大流}
注意每次清空数组的范围是 $s$ 到 $t$.
\lstinputlisting[]{./图论/dinic.cpp}
\subsubsection{最小费用最大流}
\lstinputlisting[]{./图论/费用流.cpp}
\subsubsection{最大闭权子图}
正权点向 S 连边, 负权点向 T 连边。边权为点权的绝对值。原图的边容量设为 INF 。

则最大收益为 $\sum_{v>0}v-mincost$

在最大闭权子图中的点是残量网络中 S 能到达的点。

\subsection{树哈希}
\lstinputlisting[]{./图论/TreeHash.cpp}
\subsection{强联通分量}
\lstinputlisting[]{./图论/tarjan.cpp}
\subsection{割点和桥}
\lstinputlisting[]{./图论/割点.cpp}
\subsection{点双联通分量}
\lstinputlisting[]{./图论/点双.cpp}
\subsection{边双联通分量}
\lstinputlisting[]{./图论/边双.cpp}
\subsection{2-SAT}

$2 * u$ 代表不选择,$2*u+1$ 代表选择。

也可以求强连通分量。

如果对于一个*x* `sccno`比它的反状态 *x*∧1 的 `sccno` 要小,那么我们用 *x* 这个状态当做答案,否则用它的反状态当做答案。

\lstinputlisting[]{./图论/2sat.cpp}

\section{数据结构}
\subsection{Splay}
\lstinputlisting[]{./数据结构/splay.cpp}
\subsection{李超线段树}
\begin{lstlisting}
struct Line {
	ll k, b;
} lin[N];
int lcnt;
int add_line(ll k, ll b) {
	lin[++lcnt] = {k, b};
	return lcnt;
}
struct node {
	int ls, rs, u;
} tr[N << 2];
int tot;
ll calc(int u, ll x) {
	return lin[u].k * x + lin[u].b;
}
bool cmp(int u, int v, ll x) {
	return calc(u, x) <= calc(v, x); // 如果要求最大值,只需要修改为大于等于
}
void pushdown(int &p, int l, int r, int v) {
	if (!p) p = ++tot;
	if (l == r) return;
	int mid = (l + r) >> 1;
	int &u = tr[p].u, b = cmp(v, u, mid);
	if (b) swap(u, v);
	int bl = cmp(v, u, l), br = cmp(v, u, r);
	if (bl) pushdown(tr[p].ls, l, mid, v);
	if (br) pushdown(tr[p].rs, mid + 1, r, v);
}
void update(int &p, int l, int r, int L, int R, int v) {
	if (l > R || r < L) return;
	if (!p) p = ++tot;
	int mid = (l + r) >> 1;
	if (l >= L && r <= R) return pushdown(p, l, r, v), void();
	update(tr[p].ls, l, mid, L, R, v);
	update(tr[p].rs, mid + 1, r, L, R, v);
}
ll query(int p, int l, int r, ll pos) {
	if (!p) return 1e16;
	ll res = calc(tr[p].u, pos);
	int mid = (l + r) >> 1;
	if (l == r) return res;
	if (pos <= mid) {
		res = min(res, query(tr[p].ls, l, mid, pos));
	} else res = min(res, query(tr[p].rs, mid + 1, r, pos));
	return res;
}

int main() {
	lin[0].b = 1e16;
	return 0;	
}
\end{lstlisting}
\subsection{Link Cut Tree}
\begin{lstlisting}
#include <bits/stdc++.h>
using namespace std;

const int N = 1e5 + 5;
int n, m, ch[N][2], f[N], s[N], r[N], v[N];

#define lc ch[x][0]
#define rc ch[x][1]

bool noroot(int x) {
	return ch[f[x]][1] == x || ch[f[x]][0] == x;
}
void pushup(int x) {
    s[x] = s[lc] ^ s[rc] ^ v[x];
}
void pushr(int x) {
	swap(lc, rc);
	r[x] ^= 1; 
}
void pushdown(int x) {
	if (r[x]) {
		if (lc) pushr(lc);
		if (rc) pushr(rc);
		r[x] = 0; 
	}
}
void rotate(int x) {
	int y = f[x], z = f[y], k = (ch[y][1] == x), w = ch[x][k ^ 1];
	if (noroot(y)) ch[z][y == ch[z][1]] = x;
	ch[x][k ^ 1] = y;
	ch[y][k] = w;
	f[w] = y;
	f[y] = x;
	f[x] = z;
	pushup(y), pushup(x);
}
void update(int x) {
	if (noroot(x)) update(f[x]);
	pushdown(x);
}
bool get(int x) {
	return ch[f[x]][1] == x;
}
void splay(int x) {
	update(x);
	for (int fa; fa = f[x], noroot(x); rotate(x)) {
        if (noroot(fa)) rotate(get(x) == get(fa) ? fa : x);
	}
	pushup(x);
}
void access(int x) {
    int p;
    for (p = 0; x; p = x, x = f[x]) {
        splay(x),ch[x][1] = p, pushup(x);
    }
}
void makeroot(int x) {
	access(x); splay(x);
	pushr(x);
}
int findroot(int x) {
	access(x);
	splay(x);
	while (lc) pushdown(x), x = lc;
	splay(x);
	return x;
}
void split(int x, int y) {
	makeroot(x);
    access(y);splay(y);
}
void link(int x, int y) {
	makeroot(x);
	if (findroot(y) != x) f[x] = y;
}
void cut(int x, int y) {
	makeroot(x);
	if (findroot(y) == x && f[y] == x && !ch[y][0]) {
		f[y] = ch[x][1] = 0;
		pushup(x);
	}
}

int main() {
    scanf("%d %d", &n, &m);
    for (int i = 1; i <= n; i++) scanf("%d", &v[i]);
    while (m--) {
    	int opt, x, y;
    	scanf("%d %d %d", &opt, &x, &y);
    	if (opt == 0) split(x, y), printf("%d\n", s[y]);
    	if (opt == 1) link(x, y);
    	if (opt == 2) cut(x, y);
    	if (opt == 3) splay(x), v[x] = y, pushup(x);
    }
    return 0;
}
\end{lstlisting}

\subsection{兔队线段树}

求有多少个严格前缀最大值。

线段树保存每个区间为子问题时右部分的答案 res(可以不需要信息可减),和区间的最大值 mx。

calc 考虑一段区间之前有 x 大的数时,区间此时前缀最大数的树目。

1. $x \geq val[lson],\ ans = calc(rson)$

2. $x < val[lson],\ ans = calc(lson) + res[p]$

\begin{lstlisting}
#include <bits/stdc++.h>
using namespace std;
using ll = long long;

const int N = 1e5 + 5;
#define lson (p << 1)
#define rson ((p << 1) | 1)
#define mid ((l + r) >> 1)
int n, m;
struct node {
    int s, a, b;
} tr[N << 2];
bool cmp(int a, int b, int c, int d) {
    if (d == 0 && b == 0) return 0;
    if (d == 0 && a == 0) return 0;
    if (d == 0) return 1;
    return a * 1ll * d > c * 1ll * b; 
}
int calc(int p, int l, int r, int c, int d) {
    if (l == r) 
        return cmp(tr[p].a, tr[p].b, c, d);
    if (cmp(tr[lson].a, tr[lson].b, c, d)) {
        return calc(lson, l, mid, c, d) + tr[p].s;
    }
    return calc(rson, mid + 1, r, c, d);
}
void modify(int p, int l, int r, int pos, int v) {
    if (l == r) {
        tr[p] = {0, v, pos};
        return;
    }
    if (pos <= mid) modify(lson, l, mid, pos, v);
    else modify(rson, mid + 1, r, pos, v);
    if (cmp(tr[lson].a, tr[lson].b, tr[rson].a, tr[rson].b)) {
        tr[p] = tr[lson];
    } else tr[p] = tr[rson];
    tr[p].s = calc(rson, mid + 1, r, tr[lson].a, tr[lson].b);
}

int main() {
    scanf("%d %d", &n, &m);
    while (m--) {
        int x, y;
        scanf("%d %d", &x, &y);
        modify(1, 1, n, x, y);
        printf("%d\n", calc(1, 1, n, 0, 0));
    }
    return 0;
}
\end{lstlisting}
\subsection{线段树分治}

有一个 $n$ 个节点的图。

在 $k$ 时间内有 $m$ 条边会出现后消失。

要求出每一时间段内这个图是否是二分图。

\lstinputlisting[]{./数据结构/线段树分治.cpp}

\section{字符串}
\subsection{哈希}
\subsubsection{最长回文子串}
通过哈希同样可以 $O(n)$ 解决这个问题,具体方法就是记 $R_i$ 表示以 $i$ 作为结尾的最长回文的长度,那么答案就是 $\max_{i=1}^nR_i$。考虑到 $R_i\leq R_{i-1}+2$,因此我们只需要暴力从 $R_{i-1}+2$ 开始递减,直到找到第一个回文即可。记变量 $z$ 表示当前枚举的 $R_i$,初始时为 $0$,则 $z$ 在每次 $i$ 增大的时候都会增大 $2$,之后每次暴力循环都会减少 $1$,故暴力循环最多发生 $2n$ 次,总的时间复杂度为 $O(n)$。
\subsection{字典树}
\subsection{维护异或和}
\lstinputlisting[]{./字符串/TireTree_xorsum.cpp}
\subsection{KMP}
\lstinputlisting[]{./字符串/kmp.cpp}
\subsubsection{字符串最小周期}

设 border 长度为 r

则 $s[i]=s[n-r+i]$

$|T|=n-r$
\subsubsection{每个前缀的出现次数}
1. 统计每个前缀在自身的出现次数
\begin{lstlisting}
vector<int> ans(n + 1);
for (int i = 1; i <= n; i++) ans[k[i]]++;
for (int i = n; i >= 1; i--) ans[k[i]] += ans[i];
for (int i = 1; i <= n; i++) ans[i]++;
\end{lstlisting}

2. 统计每个前缀在其他串的出现次数

我们应用来自 Knuth-Morris-Pratt 的技巧:构造一个字符串 $s + \# + t$ 并计算其前缀函数。与第一个问题唯一的不同之处在于,我们只关心与字符串 $t$ 相关的前缀函数值,即 $i \ge n + 1$ 的 $\pi[i]$。有了这些值之后,我们可以同样应用在第一个问题中的算法来解决该问题。

\subsubsection{一个字符串中本质不同子串的数目}
给定一个长度为 $n$ 的字符串 $s$,我们希望计算其本质不同子串的数目。

我们将迭代的解决该问题。换句话说,在知道了当前的本质不同子串的数目的情况下,我们要找出一种在 $s$ 末尾添加一个字符后重新计算该数目的方法。

令 $k$ 为当前 $s$ 的本质不同子串数量。我们添加一个新的字符 $c$ 至 $s$。显然,会有一些新的子串以字符 $c$ 结尾。我们希望对这些以该字符结尾且我们之前未曾遇到的子串计数。

构造字符串 $t = s + c$ 并将其反转得到字符串 $t^{\sim}$。现在我们的任务变为计算有多少 $t^{\sim}$ 的前缀未在 $t^{\sim}$ 的其余任何地方出现。如果我们计算了 $t^{\sim}$ 的前缀函数最大值 $\pi_{\max}$,那么最长的出现在 $s$ 中的前缀其长度为 $\pi_{\max}$。自然的,所有更短的前缀也出现了。

因此,当添加了一个新字符后新出现的子串数目为 $|s| + 1 - \pi_{\max}$。

所以对于每个添加的字符,我们可以在 $O(n)$ 的时间内计算新子串的数目,故最终复杂度为 $O(n^2)$。

值得注意的是,我们也可以重新计算在头部添加一个字符,或者从尾或者头移除一个字符时的本质不同子串数目。

\subsection{AC自动机}
\lstinputlisting[]{./字符串/AC.cpp}

\subsection{后缀数组}
\lstinputlisting[]{./字符串/SA.cpp}

\subsection{Manacher}
对于第 $i$ 个字符为对称轴: 

1. 如果回文串长为奇数, $d[2 * i]/2$ 是半径加上自己的长度

2. 如果长为偶数, $d[2 * i -1]/2$ 是半径的长度, 方向向右. 
\lstinputlisting[]{./字符串/Manacher.cpp}

\subsection{Z函数}
$z[i]=lcp(suf_1, suf_i)$
\lstinputlisting[]{./字符串/Z_function.cpp}
\section{数学}
\subsection{基本算法}
\lstinputlisting[]{./数学/基本算法.cpp}
\subsection{CRT}
\lstinputlisting[]{./数学/CRT.cpp}
\subsection{Lucas}
\lstinputlisting[]{./数学/Lucas.cpp}
\subsection{exLucas}
\lstinputlisting[]{./数学/exLucas.cpp}
\subsection{线性基}
\lstinputlisting[]{./数学/线性基.cpp}
\subsection{高斯消元}
\subsubsection{解线性方程组}
\lstinputlisting[]{./数学/解线性方程组.cpp}
\subsubsection{求行阶梯矩阵}
\lstinputlisting[]{./数学/求行阶梯矩阵.cpp}
\subsubsection{解不定方程}
\lstinputlisting[]{./数学/解不定方程.cpp}
\subsection{矩阵树定理}


一、无向无环图

$A$为邻接矩阵,$A[i][j]=i\to j$的边数

$D$为度数矩阵,$D[i][i]=\sum\limits_{j=1}^n A[i][j]=i$的度数,其他位置为0

基尔霍夫矩阵$K=D-A$,令$K'=K$的去掉第$k$行第$k$列($k$任意)的$n-1$阶主子式

$det(K')=$该无向图生成树个数

特别地,完全图生成树个数是$n^{n-2}$\\

二、加权

求所有生成树边权的乘积之和,需要把邻接矩阵中边的条数改为为边权和

度数矩阵改为$D[i][i]=\sum\limits_{j=1}^n A[i][j]$\\

三、有向图

对于有根外向树,需要把度数矩阵改为入度和,$D[i][i]=\sum\limits_{j=1}^nA[j][i]$

对于有根内向树,需要把度数矩阵改为出度和,$D[i][i]=\sum\limits_{j=1}^nA[i][j]$

类似地,求所有有向生成树边权的乘积之和,需要把邻接矩阵改为入/出边边权和

四、变形:边权和的和

求所有生成树边权和的和,给原先边权为$w$的边赋值为一次多项式$wx+1$,多项式乘法对$x^2$取模,$\prod (w_ix+1)$的一次项系数即为$w_i$之和

\begin{lstlisting}
struct P {
	ll x,y;//x是一次项系数, y是常数项
	P (ll x=0,ll y=0):x(x),y(y){}
	friend P operator + (const P &u, const P &v) {
		return P(add(u.x, v.x), add(u.y, v.y));
	}
	friend P operator - (const P &u, const P &v) {
		return P(add(u.x, mod - v.x), add(u.y, mod - v.y));
	}
	friend P operator * (const P &u, const P &v) {
		return P(add(mul(u.x, v.y), mul(u.y, v.x)), mul(u.y, v.y));
	}
	friend P operator / (const P &u, const P &v) {
		ll inv=qpow(v.y, mod-2);
		return P(add(mul(u.x, v.y), mod - mul(u.y, v.x)) * inv % mod * inv % mod, mul(u.y, inv));
	}
};
P g[N][N];
ll gauss(P g[N][N],int n){
	P res(0,1);
	for(int i=1;i<=n;++i) {
		int pos=-1;
		for(int j=i;j<=n;++j){
			if(g[j][i].y){
				pos=j;break;
			}
		}
		if(pos==-1)return 0;
		swap(g[i],g[pos]);
		if(pos!=i)res=res*P(0,mod-1);
		res=res*g[i][i];
		P inv=P(0,1)/g[i][i];
		for(int j=i+1;j<=n;++j){
			P t=g[j][i]*inv;
			for(int k=n;k>=i;--k)g[j][k]=g[j][k]-t*g[i][k];
		}
	}
	return res.x;
}
\end{lstlisting}
\subsection{多项式}
\subsubsection{FFT}
\lstinputlisting[]{./数学/FFT.cpp}
\subsubsection{NTT}
\lstinputlisting[]{./数学/NTT.cpp}
\subsection{组合数学}
\subsubsection{小球放盒}
第二类斯特林数(斯特林子集数)$\begin{Bmatrix}n\\ k\end{Bmatrix}$,也可记做 $S(n,k)$,表示将 $n$ 个两两不同的元素,划分为 $k$ 个互不区分的非空子集的方案数。
$$
\begin{Bmatrix}n\\ k\end{Bmatrix}=\begin{Bmatrix}n-1\\ k-1\end{Bmatrix}+k\begin{Bmatrix}n-1\\ k\end{Bmatrix}
$$

边界是 $\begin{Bmatrix}n\\ 0\end{Bmatrix}=[n=0]$。

假设小球个数为 $n$, 盒子个数为 $m$

1. 小球无标号,盒子有标号,不允许空盒。

   即求解方程 $\sum\limits_{i=1}^mx_i=n$ 解的个数

   即 $\binom{n-1}{m-1}$

2. 小球无标号,盒子有标号,允许空盒。

   令 $y_i=x_i+1$

   即求解方程 $\sum\limits_{i=1}^my_i=n$ 解的个数

   即 $\binom{n+m-1}{m-1}$

3. 小球有标号,盒子有标号,允许空盒。

   即 $m^n$

4. 小球有标号,盒子有标号,不允许空盒。

   $m!\times \begin{Bmatrix}n\\ m\end{Bmatrix}$

5. 小球有标号,盒子无标号,不允许空盒。

   $\begin{Bmatrix}n\\ m\end{Bmatrix}$

6. 小球有标号,盒子无标号,允许空盒。

   $\sum\limits_{i=1}^m\begin{Bmatrix}n\\ i\end{Bmatrix}$

7. 小球无标号,盒子无标号,允许空盒。

   设 $f[i][j]$ 表示 i 个球放入 j 个盒子的方案数。

   1. $i = 0$ 或者 $j = 1$, 方案数为 1

   2. $i < j$, $f[i][j]=f[i][i]$

   3. $i\ge j$, $f[i][j]=f[i-j][j]+f[i][j-1]$

8. 小球无标号,盒子无标号,不允许空盒。

   用 7 的结论,提前在每个盒子放 1 个球。

   方案数就是 $f[n-m][m]$
\section{MISC}
\subsection{完全平方数判断}
\lstinputlisting[]{./MISC/完全平方数判断.cpp}
\subsection{维护区间GCD值}
\lstinputlisting[]{./MISC/GCDseg.cpp}
\subsection{FastIO}
\lstinputlisting[]{./MISC/FastIO.cpp}
\end{document}
