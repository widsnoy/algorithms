\documentclass[a4paper]{article}
\usepackage[UTF8]{ctex}
\usepackage[colorlinks,
linkcolor = black,
anchorcolor = blue,
urlcolor = blue,
citecolor = blue,
]{hyperref}
\usepackage{listings}
\usepackage{xcolor}
\usepackage{booktabs} % 导入三线表需要的宏包
\usepackage{longtable}% 导入跨页表格所需宏包
\usepackage{amsmath}
\usepackage{geometry}
\usepackage{fontspec}
\usepackage{fancyhdr}
\usepackage{graphicx}
\usepackage{amsfonts}
\usepackage{chemformula}
\usepackage{cite}
\usepackage{bookmark}
\usepackage{chemformula}
\usepackage[T1]{fontenc}
\setlength{\parindent}{0pt}

\geometry{left=2.0cm,right=1.5cm,top=2.0cm,bottom=2.0cm}
\pagestyle{fancy}
\fancyhead[L]{by widsnoy}
\fancyhead[R]{\enspace}
\fancyfoot[C]{\thepage}

\begin{document}
\fontsize{12pt}{22pt}\selectfont		
\title{\Huge \textbf{ICPC TEMPLATE}}
\author{widsnoy}
\maketitle
\pagenumbering{roman}
\tableofcontents
\newpage
\pagenumbering{arabic}

	
\lstset{
	breaklines=true,
	basicstyle=\fontspec{Fira Code}\zihao{5},
	keywordstyle=\fontspec{Fira Code SemiBold}\zihao{5},
	commentstyle=\fontspec{Noto Sans CJK SC}\zihao{5},
	extendedchars=false,                        % 解决代码跨页时,章节标题,页眉等汉字不显示的问题
	backgroundcolor=\color[rgb]{0.96,0.96,0.96},
	keywordstyle=\color{blue}\bfseries,        % 关键字颜色
	commentstyle=\color{green}, %[rgb]{0,0.6,0},          % 注释颜色
	stringstyle=\color{red},                  % 字符串颜色
	showstringspaces=false,                     % 不显示字符串内的空格
	numbers=left,                               % 显示行号
	frame=single,                               % 设置代码框形式
	tabsize=4, 
	language=c++,
	captionpos=t,                               % title在上方(在bottom即为b)
	rulecolor=\color[rgb]{0.8,0.8,0.8}          % 设置代码框颜色
}  
\tableofcontents
\newpage

\section{图论}
\subsection{最近公共祖先}
\lstinputlisting[]{./图论/最近公共祖先.cpp}

\subsection{二分图}
\subsubsection*{最大匹配}
\lstinputlisting[]{./图论/二分图匹配.cpp}

\subsection{有向图最小路径覆盖问题}
\lstinputlisting[]{./图论/有向图最小路径覆盖.cpp}

\subsection{网络流}
\subsubsection{Dinic最大流}
注意每次清空数组的范围是 $s$ 到 $t$.
\lstinputlisting[]{./图论/dinic.cpp}
\subsubsection{最小费用最大流}
\lstinputlisting[]{./图论/费用流.cpp}
\subsubsection{最大闭权子图}
正权点向 S 连边, 负权点向 T 连边。边权为点权的绝对值。原图的边容量设为 INF 。

则最大收益为 $\sum_{v>0}v-mincost$

在最大闭权子图中的点是残量网络中 S 能到达的点。

\subsection{树哈希}
\lstinputlisting[]{./图论/TreeHash.cpp}
\subsection{强联通分量}
\lstinputlisting[]{./图论/tarjan.cpp}
\subsection{割点和桥}
\lstinputlisting[]{./图论/割点.cpp}
\subsection{点双联通分量}
\lstinputlisting[]{./图论/点双.cpp}
\subsection{边双联通分量}
\lstinputlisting[]{./图论/边双.cpp}
\subsection{2-SAT}

$2 * u$ 代表不选择,$2*u+1$ 代表选择。

也可以求强连通分量。

如果对于一个*x* `sccno`比它的反状态 *x*∧1 的 `sccno` 要小,那么我们用 *x* 这个状态当做答案,否则用它的反状态当做答案。

\lstinputlisting[]{./图论/2sat.cpp}

\section{字符串}
\subsection{哈希}
\subsubsection{最长回文子串}
通过哈希同样可以 $O(n)$ 解决这个问题,具体方法就是记 $R_i$ 表示以 $i$ 作为结尾的最长回文的长度,那么答案就是 $\max_{i=1}^nR_i$。考虑到 $R_i\leq R_{i-1}+2$,因此我们只需要暴力从 $R_{i-1}+2$ 开始递减,直到找到第一个回文即可。记变量 $z$ 表示当前枚举的 $R_i$,初始时为 $0$,则 $z$ 在每次 $i$ 增大的时候都会增大 $2$,之后每次暴力循环都会减少 $1$,故暴力循环最多发生 $2n$ 次,总的时间复杂度为 $O(n)$。
\subsection{字典树}
\subsection{维护异或和}
\lstinputlisting[]{./字符串/TireTree_xorsum.cpp}
\subsection{KMP}
\lstinputlisting[]{./字符串/kmp.cpp}
\subsubsection{字符串最小周期}

设 border 长度为 r

则 $s[i]=s[n-r+i]$

$|T|=n-r$
\subsubsection{每个前缀的出现次数}
1. 统计每个前缀在自身的出现次数
\begin{lstlisting}
vector<int> ans(n + 1);
for (int i = 1; i <= n; i++) ans[k[i]]++;
for (int i = n; i >= 1; i--) ans[k[i]] += ans[i];
for (int i = 1; i <= n; i++) ans[i]++;
\end{lstlisting}

2. 统计每个前缀在其他串的出现次数

我们应用来自 Knuth-Morris-Pratt 的技巧:构造一个字符串 $s + \# + t$ 并计算其前缀函数。与第一个问题唯一的不同之处在于,我们只关心与字符串 $t$ 相关的前缀函数值,即 $i \ge n + 1$ 的 $\pi[i]$。有了这些值之后,我们可以同样应用在第一个问题中的算法来解决该问题。

\subsubsection{一个字符串中本质不同子串的数目}
给定一个长度为 $n$ 的字符串 $s$,我们希望计算其本质不同子串的数目。

我们将迭代的解决该问题。换句话说,在知道了当前的本质不同子串的数目的情况下,我们要找出一种在 $s$ 末尾添加一个字符后重新计算该数目的方法。

令 $k$ 为当前 $s$ 的本质不同子串数量。我们添加一个新的字符 $c$ 至 $s$。显然,会有一些新的子串以字符 $c$ 结尾。我们希望对这些以该字符结尾且我们之前未曾遇到的子串计数。

构造字符串 $t = s + c$ 并将其反转得到字符串 $t^{\sim}$。现在我们的任务变为计算有多少 $t^{\sim}$ 的前缀未在 $t^{\sim}$ 的其余任何地方出现。如果我们计算了 $t^{\sim}$ 的前缀函数最大值 $\pi_{\max}$,那么最长的出现在 $s$ 中的前缀其长度为 $\pi_{\max}$。自然的,所有更短的前缀也出现了。

因此,当添加了一个新字符后新出现的子串数目为 $|s| + 1 - \pi_{\max}$。

所以对于每个添加的字符,我们可以在 $O(n)$ 的时间内计算新子串的数目,故最终复杂度为 $O(n^2)$。

值得注意的是,我们也可以重新计算在头部添加一个字符,或者从尾或者头移除一个字符时的本质不同子串数目。

\subsection{AC自动机}
\lstinputlisting[]{./字符串/AC.cpp}

\subsection{后缀数组}
\lstinputlisting[]{./字符串/SA.cpp}

\subsubsection{Manacher}
对于第 $i$ 个字符为对称轴: 

1. 如果回文串长为奇数, $d[2 * i]/2$ 是半径加上自己的长度

2. 如果长为偶数, $d[2 * i -1]/2$ 是半径的长度, 方向向右. 
\lstinputlisting[]{./字符串/Manacher.cpp}

\subsection{Z函数}
$z[i]=lcp(suf_1, suf_i)$
\lstinputlisting[]{./字符串/Z_function.cpp}
\section{数学}
\subsection{线性基}
\lstinputlisting[]{./数学/线性基.cpp}
\subsection{多项式}
\subsubsection{FFT}
\lstinputlisting[]{./数学/FFT.cpp}
\subsection{组合数学}
\subsubsection{小球放盒}
第二类斯特林数(斯特林子集数)$\begin{Bmatrix}n\\ k\end{Bmatrix}$,也可记做 $S(n,k)$,表示将 $n$ 个两两不同的元素,划分为 $k$ 个互不区分的非空子集的方案数。
$$
\begin{Bmatrix}n\\ k\end{Bmatrix}=\begin{Bmatrix}n-1\\ k-1\end{Bmatrix}+k\begin{Bmatrix}n-1\\ k\end{Bmatrix}
$$

边界是 $\begin{Bmatrix}n\\ 0\end{Bmatrix}=[n=0]$。

假设小球个数为 $n$, 盒子个数为 $m$

1. 小球无标号,盒子有标号,不允许空盒。

   即求解方程 $\sum\limits_{i=1}^mx_i=n$ 解的个数

   即 $\binom{n-1}{m-1}$

2. 小球无标号,盒子有标号,允许空盒。

   令 $y_i=x_i+1$

   即求解方程 $\sum\limits_{i=1}^my_i=n$ 解的个数

   即 $\binom{n+m-1}{m-1}$

3. 小球有标号,盒子有标号,允许空盒。

   即 $m^n$

4. 小球有标号,盒子有标号,不允许空盒。

   $m!\times \begin{Bmatrix}n\\ m\end{Bmatrix}$

5. 小球有标号,盒子无标号,不允许空盒。

   $\begin{Bmatrix}n\\ m\end{Bmatrix}$

6. 小球有标号,盒子无标号,允许空盒。

   $\sum\limits_{i=1}^m\begin{Bmatrix}n\\ i\end{Bmatrix}$

7. 小球无标号,盒子无标号,允许空盒。

   设 $f[i][j]$ 表示 i 个球放入 j 个盒子的方案数。

   1. $i = 0$ 或者 $j = 1$, 方案数为 1

   2. $i < j$, $f[i][j]=f[i][i]$

   3. $i\ge j$, $f[i][j]=f[i-j][j]+f[i][j-1]$

8. 小球无标号,盒子无标号,不允许空盒。

   用 7 的结论,提前在每个盒子放 1 个球。

   方案数就是 $f[n-m][m]$

\end{document}